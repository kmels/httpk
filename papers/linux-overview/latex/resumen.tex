\documentclass[spanish]{article}
\usepackage[spanish]{babel}
\usepackage[utf8x]{inputenc}

\title{
Resumen \\
Investigación 1\\
CC3008 - Sistemas Operativos Avanzados}
\author{Carlos López Camey (08107) y Héctor Hurtarte (08119)\\
Universidad del Valle de Guatemala}
\date{Septiembre del 2010}
\maketitle

\begin{document}

A continuación, un resumen de cada sección desarrollada, para más información,referirse al trabajo escrito.

\paragraph{Generalidades del diseño. } Linux es un sistema que se concentra en la estandarización, entre los que está POSIX, que define un conjunto de especificaciones. 

Compuesto de 3 partes principales:
\textbf{Kernel:} Responsable de mantener las abstracciones imortantes del S.O. e.g. memoria virtual y procesos.
\textbf{Bibliotecas:} Un conjunto estándar de funciones, la mayoría provienen del proyecto GNU.
\textbf{Utilidades del sistema:} Sus programas realizan tareas individuales y especializadas.

\paragraph{Características principales.}

Desde la perspectiva del usuario es: 
\begin{itemize}
  \item multi-tarea: varios programas pueden correr al mismo tiempo
  \item multi-usuario: varios usuarios con sesión iniciada al mismo tiempo
  \item multi-plataforma: corre en diferentes procesadores
  \item multi-procesador: soporte de multi-procesamiento simétrico para Intel y SPARC.
  \item multi-hilo: soporte para múltiples hilos de control.
\end{itemize}

Desde la perspectiva del desarrollador:
\begin{itemize}
  \item Provee protección de memoria entre procesos
  \item Memoria virtual 
  \item Librerías dinámicamente compartidas y estáticas.
\end{itemize}

\paragraph{Historia.} Linus Torvalds empezó el desarrollo del kernel en 1991, luego lo puso a disposición general compartiendolo a través de internet. De ahí en adelante, el código estubo licenciado bajo una licencia libre y por lo tanto está disponible para ser bajado.

La versión 0.01 del kernel fue liberada en mayo del 1991: solo corría sobre el procesador Intel 80386, con sistema de archivos Minix y manejo de memoria y procesos basado en Unix. 3 Años más tarde se añadiría soporte para protocolos de red como el TCP/IP, comunicación entre procesos.

\pagebreak
\paragraph{Manejo de procesos.} Basado en el estándar de POSIX, que define dos operaciones sobre los procesos:
\begin{enumerate}
  \item $fork()$: para crear un proceso
  \item $exec()$: para ejecutar un proceso
\end{enumerate}

Un proceso puede ser creado sin ser ejecutado y un proceso puede ser ejecutado sin necesidad de crear uno nuevo antes, es decir, uno podría llamar a $exec()$ sobre un proceso cualquiera y el proceso actual dejaría de ejecutarse para ejecutar la petición.

Un proceso es una entidad que abstrae toda la información que el kernel necesita para manejar su creación, ejecución, etc. Por eso define:
\begin{itemize}
  \item Identidad de un proceso: identidad asignada a cada proceso para referirse a él.
  \item Ambiente de un proceso: argumentos para la ejecución del proceso.
  \item Contexto de un proceso: información para ejecutar el proceso, como por ejemplo la prioridad de calendarización.
\end{itemize} 

\paragraph{Manejo de memoria.} Se basa en memoria virtual, por lo que el manejo de memoria se divide en:
\begin{enumerate}
  \item Memoria física: Dividida en tres zonas distintas (para dispositivos, para el procesoador y memoria para procesos del usuario)
  \item Memoria virtual: Mapea las direcciones de memoria de cada proceso en ejecución a memoria real. 
\end{enumerate}

Linux utiliza el mecanismo de paginación, en donde se decide que páginas escribir en disco, según el algoritmo \textit{second chance}, que es una combinación de FIFO y un bit de referencia, que revisa si la página ha sido referenciada.

\paragraph{Sistema de archivos.} Define cuatro tipos principales de objetos, abstrayendo sobre las operaciones de cada uno:
\begin{enumerate}
  \item Superbloque: representa un sistema de archivos completo, puede ser un sistema de archivos local o remoto.
  \item Entrada de directorio: representa a una instancia de un directorio en un superbloque.
  \item Nodo: representa un archivo individual en un superbloque.
  \item Archivo: representa un archivo abierto, para acceder a él, hay que obtener su objeto nodo primero.
\end{enumerate}

Por muchos años (1992-2001), el sistema de archivos utilizado fue el \textbf{ext2fs}, que fue reemplazado por \textbf{ext3}, que está basado en ext2 y soporta $journaling$.

\paragraph{Distribuciones.}
Muchos de los componentes de una distribución de Linux no son exclusivos de Linux. En particular, Linux utiliza muchas herramientas desarrolladas como parte de BSD de Berkeley, el sistema X Window del MIT y el proyecto GNU. 

Las distribuciones roporcionan un cojunto de paquetes estándar precompilados para realizar instalaciones Linux de manera sencilla.

Ejemplos de distribuciones:\\
\textbf{CentOS}: Distribución derivada de Red Hat, mantenida por una comunidad que tiene por objetivo sostener versiones 100\% compatibles con RedHat. \\\textbf{Debian}: Distribución no comercial mantenida por una comunidad de desarrollo comprometida con los principios del software libre. \\ \textbf{Fedora}: \textit{Community distribution} patrocinada por Red Hat. \\ \textbf{Edubuntu}: Derivado oficial de Ubuntu diseñado para uso en salones de clase, hogares y comunidades.\\ \textbf{KnoppMyth}: Una distribución diseñada para usarse con \textit{Home Theater PCs} (HTPCs). 
\paragraph{Modificaciones del kernel. }
Las modificaciones del kernel suceden rápido. De la version 2.6.20 al 2.6.25 (a\~{n}o y medio de desarrollo), \textbf{cada díaa}, en promedio 
\begin{itemize}
	\item 4300 l\'{i}neas añadidas 
	\item 1800 l\'{i}neas removidas
	\item 1500 l\'{i}neas modificadas
\end{itemize}

Existe una jerarquía de encargados que va desde desarrolladores, encargados de archivos, encargados de sub-sistemas y las personas que deciden cuando liberar una versión.

\paragraph{Market share.}\\
Computadoras personales y portatiles: Entre el 1\% y el 4\% de uso dentro de los sistemas operativos que navegaban en la web.

\textit{Netbooks}: apróximadamente un tercio (~34\%) del mercado, en 2009 se estima que se habían vendido cerca de 35 millones de $netbooks$ en total. 

Servidores:  En el primer cuarto del 2010, la IDC (\texit{International Data Corporation} publicó que apróximadamente el 20\% del mercado pertenecía a Linux.  Otras fuentes dicen que el verdadero porcentaje es de 40\%.

Mainframes: En 2010, totalmente dominado por el Sistema z de IBM, aunque Red Hat alegaba el 18.4\% en 2007 y el 37\% en 2008.

Supercomputadoras: Dominado por Linux, en un 91\%, seguido por Unix (4.40\%).

\end{document}
